\documentclass[11pt,a4paper,sans]{moderncv}
\usepackage{fontawesome5}
\usepackage{xcolor}

\moderncvcolor{black}  
% ==== Style & colors ====
\moderncvstyle{banking}         % 'classic', 'banking', 'casual', 'oldstyle', 'fancy'
         % 'blue', 'green', 'orange', 'red', 'purple', 'grey', 'black'
\nopagenumbers{}                % cleaner for 1-page CV




% ==== Encoding & geometry ====
\usepackage[utf8]{inputenc}
\usepackage[scale=0.9]{geometry}      % play with scale to fit 1 page
\setlength{\hintscolumnwidth}{2.5cm}   % date column width
\usepackage{tikz}

\usepackage{tabularx}
\usepackage{ragged2e}

\newcommand{\projecttitle}[1]{\textbf{\large\textcolor{mysection}{#1}}}


\makeatletter
 \definecolor{myproject}{RGB}{184,134,11} % dark goldenrod
 % \definecolor{myproject}{RGB}{72,60,50} % taupe
% \definecolor{myproject}{RGB}{54,69,79} %charcoal
% \definecolor{myproject}{RGB}{95,158,160}   % cadet blue
 \definecolor{mysection}{RGB}{95,158,160}   % cadet blue
% \definecolor{mysection}{RGB}{143,188,143} % sage green
% \definecolor{mysection}{RGB}{189,183,107} % khaki
 %\definecolor{mysection}{RGB}{54,69,79} charcoal
% \definecolor{mysection}{RGB}{112,128,144} % slate gray
% \definecolor{mysection}{RGB}{72,60,50} % taupe




\makeatletter
\colorlet{sectioncolor}{mysection}
\colorlet{subsectioncolor}{mysection}
\colorlet{bodyrulecolor}{mysection}   % horizontal rules in 'banking'
\makeatother
\colorlet{color1}{myproject}


\usepackage{paracol}     % for two independent columns
\setlength{\columnsep}{1.2cm} % gap between the two columns
\usepackage[T1]{fontenc} % pentru bold în \texttt
\usepackage{lmodern}
% mărime uniformă cu restul corpului (folosește ce ai în \cvbody)
\newcommand{\techstack}[1]{%
  {\cvbody\ttfamily\bfseries \faTools\ Tech~Stack:}~{\cvbody\ttfamily\bfseries #1}%
}

% ==== Tighten section spacing (moderncv) ====
\usepackage{etoolbox}
\makeatletter
% If your moderncv defines \sectionvspace, shrink it; otherwise patch the formatter.
\@ifundefined{sectionvspace}{}{
  \renewcommand*{\sectionvspace}{2ex}%
}
% Reduce vertical space after section headings (works across moderncv styles)
\patchcmd{\sectionlinesformat}
  {\par\nobreak\vspace*{2.5ex}}
  {\par\nobreak\vspace*{1.0ex}}
  {}{}
\makeatother
\newcommand{\myemail}{alexandru.dabu123@gmail.com}
\newcommand{\myphone}{+40 756 517 830}
\newcommand{\mycity}{Bucharest, Romania}
\newcommand{\mygithub}{AnduDabu}
\newcommand{\mylinkedin}{alexandru-dabu}

% ==== Personal info ====
\name{Alexandru}{Dabu}
\address{Bucharest, Romania}{}
\phone[mobile]{+40 756517830}
\email{alexandru.dabu123@gmail.com}
\social[github]{AnduDabu}
\social[linkedin]{alexandru-dabu}
% \homepage{<portfolio-or-app-link>}
\usepackage[hidelinks]{hyperref}
\usepackage{ifthen}       % already used above
\usepackage{fontawesome5} % for icons

\makeatletter
% ====== Mărimea fontului pentru TOT conținutul de sub secțiuni ======
%  \small / \normalsize / \footnotesize
\newcommand{\cvbody}{\small}
    
% Redefinim \cvitem și \cventry ca să aplice \cvbody pe conținut
\let\oldcvitem\cvitem
\renewcommand{\cvitem}[2]{\oldcvitem{#1}{{\cvbody #2}}}

\let\oldcventry\cventry
\renewcommand{\cventry}[6]{\oldcventry{#1}{#2}{#3}{#4}{#5}{{\cvbody #6}}}

% Asigurăm aceeași mărime și în liste
\AtBeginEnvironment{itemize}{\cvbody}
\AtBeginEnvironment{enumerate}{\cvbody}
\AtBeginEnvironment{description}{\cvbody}

% ====== HEADER ======
\renewcommand*{\makecvtitle}{%
  \vspace*{-5mm}
  \begin{minipage}[c]{0.2\textwidth}
  \centering
  \begin{tikzpicture}
    % crop to a circle
    \clip (0,0) circle (0.45\linewidth); 
    \node at (0,0) {\includegraphics[width=0.9\linewidth]{pozaCV.jpeg}};
    % optional thin border; pick your color (mysection / myproject / black)
    \draw[line width=0.8pt, color=mysection] (0,0) circle (0.45\linewidth);
  \end{tikzpicture}
\end{minipage}\hfill
  \begin{minipage}[c]{0.79\textwidth}
    {\LARGE\bfseries \@firstname~\@lastname}\par
    \vspace{0.6ex}
    \ifthenelse{\equal{\@title}{}}{}{{\large \@title}\par}
    \vspace{0.8ex}
    {\small 
    \renewcommand{\arraystretch}{1.15}%
    \setlength{\tabcolsep}{0pt}%
    \begin{tabularx}{\linewidth}{@{}>{\hsize=0.08\hsize}l @{\hspace{0.6em}} >{\RaggedRight\arraybackslash}X@{}}
      \raisebox{0.2ex}{\faMapMarker*} & \mycity \\
      \faPhone                         & \myphone \\
      \faEnvelope                      & \href{mailto:\myemail}{\myemail} \\
      \faLinkedin                      & \href{https://www.linkedin.com/in/\mylinkedin}{linkedin.com/in/\mylinkedin} \\
      \faGithub                        & \href{https://github.com/\mygithub}{github.com/\mygithub} \\
      \faGlobe                         & Romanian — native \textbullet\ English — fluent \\
    \end{tabularx}}%
  \end{minipage}
  \par\vspace{2ex}
}
\makeatother

\begin{document}

\makecvtitle
\vspace*{-4ex} % tighten after title

\vspace{-0.2em}
% ==== Summary ====
\section{\faIdCard\ Summary}
\vspace{-0.5em}
\cvitem{}{
Recent graduate in \textbf{Automation and Computer Science} with hands-on experience ranging from hardware, software, low-level programming, and robotics to advanced control design, AI/ML-driven automation, and cross-platform mobile development. Passionate about exploring new technologies and applying technical skills to solve real-world challenges.
}
\vspace{-1.4em}
\section{\faGraduationCap\ Education}
\vspace{-0.5em}
\cvitem{2025--Present}{Master's Degree in Information Systems and Digital Transformation in Materials Processing, 
\textbf{University Politehnica of Bucharest}, Faculty of Materials Science and Engineering}
\vspace{-0.1em}
\cvitem{2021--2025}{Bachelor's Degree in Automatic Control and Computer Engineering, 
\textbf{University Politehnica of Bucharest}, Faculty of Automatic Control and Computer Engineering}
\vspace{-0.1em}
\cvitem{2017--2021}{High School Diploma, \textbf{Colegiul Național „Radu Greceanu”}, Slatina — Mathematics \& Informatics Specialization}


\vspace{-1.4em}


% ==== Projects ====
\section{\faProjectDiagram\ Projects}

\vspace{-0.3em}
\cventry{2025}{\projecttitle{\faBasketballBall\ Basketball Social Media \& AI Shot Analysis App (B.Sc. Thesis)}}{}{}{}{
The project provided real-time computer vision for basketball shot evaluation, complemented by features that foster player interaction and feedback sharing.
\begin{itemize}\itemsep0.05em
  \item \textbf{Designed and implemented a cross-platform application} combining social networking with AI-powered basketball shot analysis.
  \item \textbf{Enabled engagement features} including accounts, personal profiles, posts, comments, private messaging, and notifications.
  \item \textbf{Delivered automated performance feedback} by building a YOLOv8-based shot detection pipeline with pose estimation and trajectory heuristics.
  \item \textbf{Optimized inference pipeline} to run \textbf{real-time video analysis at 30 FPS on GPU}, with fallback to CPU for broader compatibility.
  \item \textbf{Improved training usability} through interactive maps of basketball courts and an integrated chatbot for tactical advice.
\end{itemize}
 \techstack{Python(YOLOv8, OpenCV, Flask), Flutter, Firebase, REST, Google Maps API, OpenAI API}
}

\cventry{2025}{\projecttitle{\faSitemap\ Path Planning in Radioactive Environment using Probabilistic Roadmaps}}{}{}{}{
The project focused on building a customizable, safe and efficient robot navigation in dangerous environments.
\begin{itemize}\itemsep0.05em
  \item \textbf{Built a path planning framework} using Probabilistic Roadmaps (PRM) for efficient navigation in complex, static obstacle environments.
  \item \textbf{Designed interactive tools}, including a simulator and visualization interfaces, enabling experimentation and analysis of pathfinding behavior.
  \item \textbf{Improved safety and robustness} by implementing obstacle inflation, safety margins, and adjustable connection thresholds, validated across diverse maps.
\end{itemize}
\techstack{Python, NumPy, Tkinter, Matplotlib, PIL}
}

\cventry{2025}{\projecttitle{\faRocket\ Multi-Agent Formation Control}}{}{}{}{
The project aimed to create a system aimed at enabling coordinated movement of robot teams while preserving a stable formations.
\begin{itemize}\itemsep0.05em
  \item \textbf{Applied advanced control methods} by combining graph Laplacians, control barrier functions, and spiking neural networks.
  \item \textbf{Increased robustness} by avoiding collisions through distributed control strategies and dynamically changing formations (line,
square, diamond, trapezoid).
  \item \textbf{Explored scalability and performance trade-offs} with trajectory visualizations, scalability and stability plots for effective evaluation.
\end{itemize}
\techstack{Python, NumPy, SciPy, Matplotlib, PyTorch}
}

\enlargethispage{3\baselineskip}
\cventry{2023}{\projecttitle{\faCloudRain\ Rain Prediction in Australia}}{}{}{}{
The project investigated how machine learning can leverage historical weather data to improve short-term rainfall predictions and model interpretability.
\begin{itemize}\itemsep0.05em
  \item \textbf{Achieved 80\% prediction accuracy} by developing an end-to-end ML pipeline to forecast next-day rainfall using a dataset of 140k+ records.
  \item \textbf{Improved model accuracy by 12\%} compared to logistic regression baseline using Random Forests.
  \item \textbf{Reduced false negatives by 18\%} through hyperparameter tuning, improving recall for rainy days.
  \item \textbf{Conducted exploratory data analysis} to identify high-impact meteorological features correlated with rainfall, strengthening model transparency and explainability.
\end{itemize}
\techstack{Python, Pandas, scikit-learn, Matplotlib, Jupyter}
}

\newpage
\cventry{2024}{\projecttitle{\faLock\ ARP Spoofing \& Man-in-the-Middle}}{}{}{}{
The project demonstrated network vulnerabilities by implementing ARP spoofing in a controlled lab environment and exploring defenses.
\begin{itemize}\itemsep0.05em
  \item \textbf{Developed packet manipulation routines} for interception, modification, and forwarding of packets.
  \item \textbf{Simulated ARP spoofing attacks} to show how attackers can intercept and manipulate network traffic.
  \item \textbf{Exposed security risks} such as credential theft, silent monitoring, and session hijacking.
  \item \textbf{Documented countermeasures} including encrypted protocols, static ARP entries, and inspection tools.
  \item \textbf{Strengthened networking expertise} through analysis of ARP/TCP traffic with Wireshark.
\end{itemize}
\techstack{Python, Scapy, Wireshark, Kali Linux}
}

\cventry{2023}{\projecttitle{\faMicrochip\ Image Processing Pipeline in Verilog}}{}{}{}{
The project implemented a hardware image processing pipeline on FPGA using finite state machines and memory management techniques.
\begin{itemize}\itemsep0.05em
  \item \textbf{Implemented a hardware pipeline} performing flip, grayscale, and sharpen operations on FPGA.
  \item \textbf{Ensured synchronization} with FSMs, registers, and counters coordinating pixel processing cycles.
  \item \textbf{Balanced performance and resource use} with sequential and parallel logic optimized for throughput.
\end{itemize}
\techstack{Verilog, FSM, FPGA tools, Xilinx
}}

\cventry{2023}{\projecttitle{\faShoppingBag\ E-commerce Management System}}{}{}{}{
The project provided a prototype of an online store, from catalog management to checkout and returns, built with structured software engineering methods.
\begin{itemize}\itemsep0.05em
  \item \textbf{Delivered core features of an e-commerce platform}, including product catalog, cart management, and order lifecycle.
  \item \textbf{Driven team collaboration} with UML diagrams for architecture, user flows, and business processes.
  \item \textbf{Strengthened testing coverage} with unit tests for order, cart, and return functionality.
\end{itemize}
\techstack{Java, Spring Boot (intro), JUnit, UML}
}

\vspace{-1.2em}
% ==== Work Experience ====

\section{\faBriefcase\ Work Experience}
\vspace{-0.8em}
\cventry{}% 1: (gol)
{%
  \begin{tabular*}{\linewidth}{@{}l @{\extracolsep{\fill}} r@{}}
    \textbf{Carpasoft — Mobile Developer Intern} & \textit{Jul 2024 -- Oct 2024} \\
  \end{tabular*}%
}% 2: titlu + data (în același rând)
{}% 3: (gol)
{}% 4: (gol)  — Bucharest eliminat
{}% 5: (gol)
{% 6: descriere
The internship focused on building a cross-platform mobile application for pet-related management services.
\begin{itemize}\itemsep0.05em
  \item \textbf{Extended user and pet profiles} with editable attributes and health records.
  \item \textbf{Developed booking and scheduling features} for veterinary visits, grooming, and pet walking services.
  \item \textbf{Improved community features} such as posts, stories, and real-time chat to connect users and providers.
  \item \textbf{Integrated maps and geolocation} for discovering pet-related facilities and tracking lost pets.
  \item \textbf{Improved software quality} by implementing automated test suites within CI/CD pipelines.
\end{itemize}
\techstack{Flutter, Google Cloud Services, Maps/Geolocation, GitLab, Linux, Android, iOS
}}




\end{document}
